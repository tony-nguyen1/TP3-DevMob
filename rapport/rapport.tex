% Une ligne commentaire débute par le caractère « % »

\documentclass[a4paper]{article}

% Options possibles : 10pt, 11pt, 12pt (taille de la fonte)
%                     oneside, twoside (recto simple, recto-verso)
%                     draft, final (stade de développement)

\usepackage[utf8]{inputenc}   % LaTeX, comprends les accents !
\usepackage[T1]{fontenc}      % Police contenant les caractères français
\usepackage[francais]{babel}
\usepackage{fullpage}
\usepackage{multicol}
\usepackage{hyperref}
\hypersetup{
    colorlinks=true,
    linkcolor=blue,
    filecolor=magenta,      
    urlcolor=red
    }
% \urlstyle{same} % ça sert à rien ce truc
\usepackage{bookmark}
\usepackage{blindtext}



\usepackage{graphicx}  % pour inclure des images
\graphicspath{ {rapport/img/} }

%\pagestyle{headings}        % Pour mettre des entêtes avec les titres
                              % des sections en haut de page

\title{  TP2 : Les capteurs\\Programmation mobile}
\author{Mohamad Satea Almallouhi - Tony Nguyen\\\emph{M1 Génie Logiciel}\\Faculté des Sciences\\Université de Montpellier.}
\date{5 mars 2024}



\begin{document}
    \maketitle
    \begin{center}
        \includegraphics[height=.95\textwidth]{power}
    \end{center}

    \begin{abstract}     % Résumé du travail
      \emph{Nous avons réalisé une application Android en Java afin de démontrer l'utilisation des capteurs intégré.}
    \end{abstract}
    \newpage
    %\dominitoc  % initializer les minitoc
    \tableofcontents

    \newpage
    \begin{multicols}{2}
        % [
        %     Faire une vidéo, le rapport avec des screenshot des résultats et du code et enfin un read.md(instruction). En plus, pour le bonus, faire une belle application, des tests unitaires, utiliser Kotlin, faire le rapport en Latex.
        % ]
        \section*{Introduction}
        \addcontentsline{toc}{section}{Introduction}
        \section*{Démonstration vidéo}
        \addcontentsline{toc}{section}{Démonstration}

        \section{Fragment}
            Work In Progress - Almost Done
            \subsection{Création}
            \subsection{Couper l'écran en 2}
                C'était trop dur.
            \subsection{Communication entre fragment}
            \subsection{La synchronisation}
        \section{Persistance}
            \subsection{Persistance}
            \subsection{Saisie automatique}
        \section{Réseau}
            Cet exerice n'a pas été réalisé.
        \section{Service}
        \section{Bonus: LifeCycle}
        \section{Bonus: ViewModel}
        \section{Bonus: LiveData}
        \section{Bonus: Room}

    \end{multicols}
\end{document}